% vim:ft=tex:

\section{Proyectos}

\begin{entrylist}
  \entry
  {2016}
  {Universidad de Santiago de Chile}
  {Santiago, Chile}
  {\jobtitle{Aplicativo web para encuestas 360}\\
  El Manejador Universal de Encuestas Libres (MANUEL) era un proyecto tanto 
  para el profesor como para los alumnos. Su función era calificar a los 
  miembros de un grupo de trabajo en base a la calificación de sus pares como 
  también la autoevaluación.\\
  \begin{description}
    \item[Ruby on Rails] para el backend
    \item[Angular JS] frontend
    \item[GIT] versionamiento
    \item[Materialize] framework CSS
  \end{description}
  }

  \entry
  {2016}
  {Universidad de Santiago de Chile}
  {Santiago, Chile}
  {\jobtitle{Aplicación web para la FEUSACH}\\
  La aplicación web era una herramienta para que los miembros de la federación 
  de estudiantes subiera las actas e información relevante, como la asistencia 
  y el voto por carrera, luego de hacer una asamblea. El proyecto lo hicimos 
  en Java Empresarial junto con Primefaces como framework para la interfaz de 
  usuario.\\
  \begin{description}
    \item[Java EE] para el framework MVC
    \item[Primefaces] como framework UI
    \item[GlassFish] como servidor de aplicaciones
    \item[NetBeans] como IDE
    \item[GIT] para el versionamiento
    \item[Scrum] como metodología de desarrollo
    \item[PostgreSQL] como motod de base de datos
  \end{description}}

  \entry
  {2015}
  {Universidad de Santiago de Chile}
  {Santiago, Chile}
  {\jobtitle{CRM para un hotel de mascotas}\\
  Levantamiento de procesos, modelamiento de la BD y creación de la plataforma 
  web para agendar y gestionar la estadía de las mascotas en este hotel. Para 
  este proyecto se usó RUP como metodología de desarrollo, haciendo una 
  entrega cada 2 semanas.\\
  \begin{description}
    \item[Ruby on Rails] como framework MVC
    \item[RUP] (proceso unificado racional) como metodología de desarrollo
    \item[GIT] versionamiento
    \item[PostgreSQL] como motor de base de datos
  \end{description}
  }
\end{entrylist}
