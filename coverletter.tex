%!TEX TS-program = xelatex
%!TEX encoding = UTF-8 Unicode
% Awesome CV LaTeX Template for Cover Letter
%
% This template has been downloaded from:
% https://github.com/posquit0/Awesome-CV
%
% Authors:
% Claud D. Park <posquit0.bj@gmail.com>
% Lars Richter <mail@ayeks.de>
%
% Template license:
% CC BY-SA 4.0 (https://creativecommons.org/licenses/by-sa/4.0/)
%


%-------------------------------------------------------------------------------
% CONFIGURATIONS
%-------------------------------------------------------------------------------
% A4 paper size by default, use 'letterpaper' for US letter
\documentclass[11pt, spanish, letterpaper]{awesome-cv}

% language
\usepackage{polyglossia}
\setmainlanguage{spanish}

% Configure page margins with geometry
\geometry{left=1.4cm, top=.8cm, right=1.4cm, bottom=1.8cm, footskip=.5cm}

% Specify the location of the included fonts
\fontdir[fonts/]

% Color for highlights
% Awesome Colors: awesome-emerald, awesome-skyblue, awesome-red, awesome-pink, awesome-orange
%                 awesome-nephritis, awesome-concrete, awesome-darknight
\colorlet{awesome}{awesome-red}
% Uncomment if you would like to specify your own color
% \definecolor{awesome}{HTML}{CA63A8}

% Colors for text
% Uncomment if you would like to specify your own color
% \definecolor{darktext}{HTML}{414141}
% \definecolor{text}{HTML}{333333}
% \definecolor{graytext}{HTML}{5D5D5D}
% \definecolor{lighttext}{HTML}{999999}

% Set false if you don't want to highlight section with awesome color
\setbool{acvSectionColorHighlight}{true}

% If you would like to change the social information separator from a pipe (|) to something else
\renewcommand{\acvHeaderSocialSep}{\quad\textbar\quad}


%-------------------------------------------------------------------------------
%	PERSONAL INFORMATION
%	Comment any of the lines below if they are not required
%-------------------------------------------------------------------------------
% Available options: circle|rectangle,edge/noedge,left/right
% \photo[circle,noedge,left]{profile}
\name{Ian D.}{Mejias}
\position{Desarrollador Full Stack}
\address{Victoria 349, Quilpué, Región de Valparaíso.}

\mobile{(+56) 95447 1278}
\email{idmconejeros@gmail.com}
% \homepage{www.posquit0.com}
\github{bangaa}
% \linkedin{posquit0}
% \gitlab{gitlab-id}
% \stackoverflow{SO-id}{SO-name}
% \twitter{@twit}
% \skype{skype-id}
% \reddit{reddit-id}
% \extrainfo{extra informations}

\quote{``La persona sabia (o virtuosa) es aquella que sabe lo que es bueno y 
que espontáneamente lo realiza."}


%-------------------------------------------------------------------------------
%	LETTER INFORMATION
%	All of the below lines must be filled out
%-------------------------------------------------------------------------------
% The company being applied to
\recipient
  {Vicerrectoría Académica}
  {Dirección General de Docencia\\Av. España 1680\\Valparaíso}
% The date on the letter, default is the date of compilation
\letterdate{\today}
% The title of the letter
\lettertitle{Aplicación de Ingreso a nivel intermedio}
% How the letter is opened
\letteropening{Estimado/a Responsable de aceptación}
% How the letter is closed
\letterclosing{Sinceramente,}
% Any enclosures with the letter
% \letterenclosure[Attached]{Curriculum Vitae}


%-------------------------------------------------------------------------------
\begin{document}

% Print the header with above personal informations
% Give optional argument to change alignment(C: center, L: left, R: right)
\makecvheader[C]

% Print the footer with 3 arguments(<left>, <center>, <right>)
% Leave any of these blank if they are not needed
\makecvfooter
  {\today}
  {Ian D. Mejias~~~·~~~Carta de presentación}
  {}

% Print the title with above letter informations
\makelettertitle

%-------------------------------------------------------------------------------
%	LETTER CONTENT
%-------------------------------------------------------------------------------
\begin{cvletter}

\lettersection{Acerca de mi}

Mi nombre es Ian y tengo 27 años, de los cuáles 6 años fueron como estudiante 
  de la Universidad de Santiago de Chile en la carrera Ingeniería Civil en 
  Informática. En la Universidad fui ayudante por más de 3 años en diversas 
  asignaturas y con diversas responsabilidades; no sólo porque me ayudaba a 
  pagar mi locomoción y material de estudio sino porque me gustaba hacerlo y 
  era bueno haciéndolo. Me destaqué en varias asignaturas y me hice reconocer 
  por varios profesores debido a mis habilidades, mi asertividad y mi buena 
  educación, pero dentro de todo fui un estudiante promedio pero esforzado.

  En el año 2016, debido a una serie de eventos desafortunados y sumándole a 
  eso mi personalidad, me dio una depresión crónica que no me permitió seguir 
  con mis estudios. Congelé la Universidad 3 semestres (presentando los 
  respectivos certificados médicos) pero luego del último me comunicaron que 
  no podía volver a hacerlo y por no estar ``listo" para volver a la 
  Universidad simplemente no volví a matricularme.

  Escribo esta carta para ser considerado como futuro estudiante y para pedir 
  la ayuda necesaria para terminar mi carrera en la que estuve tan cerca de 
  titularme.

\lettersection{¿Por qué USM?}

Porque cuando pienso en estudiar Ingeniería las únicas Universidades que se me 
  vienen a la mente son la USACH y la Universidad Federico Santa María.  
  Siento que puedo seguir creciendo en esta Universidad y que me puede 
  entregar mucho, y yo espero que con todo lo que he aprendido pueda 
  entregarle algo también.

\lettersection{¿Por qué yo?}

  Siento que estoy en el {\em peak} de mi vida, soy inteligente y capaz, 
  creo ser moralmente y éticamente correcto y me esfuerzo en ser la mejor 
  versión de mi todo el tiempo; siempre buscando alcanzar la maestría ética.  
  Luego de cambiarme de domicilio 3 veces (después del incidente) mi vida se 
  ordenó y estabilizó haciendome madurar como ser humano.

\end{cvletter}


%-------------------------------------------------------------------------------
% Print the signature and enclosures with above letter informations
\makeletterclosing

\end{document}
